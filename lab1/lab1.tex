\documentclass[12pt]{article}

\usepackage{fullpage}
\usepackage{multicol,multirow}
\usepackage{tabularx}
\usepackage{ulem}
\usepackage[utf8]{inputenc}
\usepackage[russian]{babel}
\usepackage{minted}

\usepackage{color} %% это для отображения цвета в коде
\usepackage{listings} %% собственно, это и есть пакет listings

\lstset{ %
language=C++,                 % выбор языка для подсветки (здесь это С++)
basicstyle=\small\sffamily, % размер и начертание шрифта для подсветки кода
numbers=left,               % где поставить нумерацию строк (слева\справа)
%numberstyle=\tiny,           % размер шрифта для номеров строк
stepnumber=1,                   % размер шага между двумя номерами строк
numbersep=5pt,                % как далеко отстоят номера строк от подсвечиваемого кода
backgroundcolor=\color{white}, % цвет фона подсветки - используем \usepackage{color}
showspaces=false,            % показывать или нет пробелы специальными отступами
showstringspaces=false,      % показывать или нет пробелы в строках
showtabs=false,             % показывать или нет табуляцию в строках
frame=single,              % рисовать рамку вокруг кода
tabsize=2,                 % размер табуляции по умолчанию равен 2 пробелам
captionpos=t,              % позиция заголовка вверху [t] или внизу [b] 
breaklines=true,           % автоматически переносить строки (да\нет)
breakatwhitespace=false, % переносить строки только если есть пробел
escapeinside={\%*}{*)}   % если нужно добавить комментарии в коде
}


\begin{document}
\begin{titlepage}
\begin{center}
\textbf{МИНИСТЕРСТВО ОБРАЗОВАНИЯ И НАУКИ РОССИЙСКОЙ ФЕДЕРАЦИИ
\medskip
МОСКОВСКИЙ АВИАЦИОННЫЙ ИНСТИТУТ
(НАЦИОНАЛЬНЫЙ ИССЛЕДОВАТЕЛЬСКИЙ УНИВЕРСТИТЕТ)
\vfill\vfill
{\Huge ЛАБОРАТОРНАЯ РАБОТА №1} \\
по курсу объектно-ориентированное программирование
I семестр, 2021/22 уч. год}
\end{center}
\vfill

Студент \uline{\it {Пономарев Никита Владимирович, группа М8О-207Б-20}\hfill}

Преподаватель \uline{\it {Дорохов Евгений Павлович}\hfill}

\vfill
\end{titlepage}

\subsection*{Условие}

Задание: \
Вариант 19: Прямоугольник, трапеция, ромб.\
Необходимо спроектировать и запрограммировать на языке C++ классы трех фигур, согласно варианту задания. Классы должны удовлетворять следующим правилам:
\begin{enumerate}
\item Должны быть названы также, как в вариантах задания и расположенны в раздельных файлах: отдельно заголовки (имя\_класса\_с\_маленькой\_буквы.h), отдельно описание методов (имя\_класса\_с\_маленькой\_буквы.cpp).
\item Иметь общий родительский класс Figure;
\item Содержать конструктор, принимающий координаты вершин фигуры из стандартного потока std::cin, расположенных через пробел. Пример: "0.0 0.0 1.0 0.0 1.0 1.0 0.0 1.0"
\item Содержать набор общих методов:
\begin{itemize}
    \item size\_t VertexesNumber() - метод, возвращающий количество вершин фигуры;
    \item double Area() - метод расчета площади фигуры;
    \item void Print(std::ostream& os) - метод печати типа фигуры и ее координат вершин в поток вывода os в формате: "Rectangle: (0.0, 0.0) (1.0, 0.0) (1.0, 1.0) (0.0, 1.0)" с переводом строки в конце.
\end{itemize}
\end{enumerate}

\subsection*{Описание программы}

Исходный код лежит в 11 файлах:
\begin{enumerate}
\item main.cpp: основная программа, взаимодействие с пользователем посредством комманд из меню

\item figure.h:    описание абстрактного класса фигур

\item point.h:     описание класса точки
\item rectangle.h:  описание класса прямоугольника, наследующегося от figures
\item rhombus.h: описание класса ромба, наследующегося от figures
\item trapezoid.h:    описание класса трапеции, наследующегося от figure

\item point.cpp:     реализация класса точки

\item rectangle.cpp: реализация класса прямоугольника, наследующегося от figures
\item rhombus.cpp:  реализация класса ромба, наследующегося от figures
\item trapezoid.cpp:    реализация класса трапеции, наследующегося от figure

\end{enumerate}

\subsection*{Дневник отладки}
Возникли проблемы при вычислении площади трапеции. В моей программе была использована формула, вычисляющая площадь как произведение диагоналей на половину косинуса угла между ними. Из за неправильной методики находения косинуса, площадь вычислялась неверно. Этот недочет удалось заметить при тестировании и, изменив формулу рассчета косинуса на более общую, и исправить.

\subsection*{Недочёты}


\subsection*{Выводы}
В процессе выполнения работы я на практике познакомился с принципами ООП, реализовал несколько классов данных(фигуры), и для каждого из них - функции. Научился перегружать операторы для более комфортной работы с моими классами.


\vfill
\pagebreak
\subsection*{Исходный код:}

{\Huge figure.h}
\inputminted
    {C++}{figure.h}
    
{\Huge point.h}
\inputminted
    {C++}{point.h}
    
{\Huge point.cpp}
\inputminted
    {C++}{point.cpp}

{\Huge rectangle.h}
\inputminted
    {C++}{rectangle.h}
    
{\Huge rectangle.cpp}
\inputminted
    {C++}{rectangle.cpp}

{\Huge rhombus.h}
\inputminted
    {C++}{rhombus.h}
    
{\Huge rhombus.cpp}
\inputminted
    {C++}{rhombus.cpp}

{\Huge trapezoid.h}
\inputminted
    {C++}{trapezoid.h}
    
{\Huge trapezoid.cpp}
\inputminted
    {C++}{trapezoid.cpp}

{\Huge main.cpp}
\inputminted
    {C++}{main.cpp}
    \pagebreak
    
\end{document}
