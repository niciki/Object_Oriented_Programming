\documentclass[12pt]{article}

\usepackage{fullpage}
\usepackage{multicol,multirow}
\usepackage{tabularx}
\usepackage{ulem}
\usepackage[utf8]{inputenc}
\usepackage[russian]{babel}
\usepackage{minted}

\usepackage{color} %% это для отображения цвета в коде
\usepackage{listings} %% собственно, это и есть пакет listings

\lstset{ %
language=C++,                 % выбор языка для подсветки (здесь это С++)
basicstyle=\small\sffamily, % размер и начертание шрифта для подсветки кода
numbers=left,               % где поставить нумерацию строк (слева\справа)
%numberstyle=\tiny,           % размер шрифта для номеров строк
stepnumber=1,                   % размер шага между двумя номерами строк
numbersep=5pt,                % как далеко отстоят номера строк от подсвечиваемого кода
backgroundcolor=\color{white}, % цвет фона подсветки - используем \usepackage{color}
showspaces=false,            % показывать или нет пробелы специальными отступами
showstringspaces=false,      % показывать или нет пробелы в строках
showtabs=false,             % показывать или нет табуляцию в строках
frame=single,              % рисовать рамку вокруг кода
tabsize=2,                 % размер табуляции по умолчанию равен 2 пробелам
captionpos=t,              % позиция заголовка вверху [t] или внизу [b] 
breaklines=true,           % автоматически переносить строки (да\нет)
breakatwhitespace=false, % переносить строки только если есть пробел
escapeinside={\%*}{*)}   % если нужно добавить комментарии в коде
}


\begin{document}
\begin{titlepage}
\begin{center}
\textbf{МИНИСТЕРСТВО ОБРАЗОВАНИЯ И НАУКИ РОССИЙСКОЙ ФЕДЕРАЦИИ
\medskip
МОСКОВСКИЙ АВИАЦИОННЫЙ ИНСТИТУТ
(НАЦИОНАЛЬНЫЙ ИССЛЕДОВАТЕЛЬСКИЙ УНИВЕРСТИТЕТ)
\vfill\vfill
{\Huge ЛАБОРАТОРНАЯ РАБОТА №2} \\
по курсу объектно-ориентированное программирование
I семестр, 2021/22 уч. год}
\end{center}
\vfill

Студент \uline{\it {Пономарев Никита Владимирович, группа М8О-207Б-20}\hfill}

Преподаватель \uline{\it {Дорохов Евгений Павлович}\hfill}

\vfill
\end{titlepage}

\subsection*{Условие}
Создать класс Address для работы с адресами домов. Адрес должен состоять из 
строк с названием города и улицы и чисел с номером дома и квартиры. Реализовать 
операции сравнения адресов, а также операции проверки принадлежности адреса к 
улице и городу. В операциях не должен учитываться регистр строки. Так же 
необходимо сделать операцию, которая возвращает истину если два адреса 
находятся по соседству (на одной улице в одном городе и дома стоят подряд). \\
Реализовать над объектами реализовать в виде перегрузки операторов.
Реализовать пользовательский литерал для работы с константами объектов
созданного класс
Исходный код лежит в 3 файлах:
\begin{enumerate}
\item main.cpp: основная программа, взаимодействие с пользователем посредством команд из меню
\item adress.h:    описание класса адресов
\item adress.cpp:  реализация класса адреса

\end{enumerate}
\pagebreak
\subsection*{Протокол работы}
(Moscow, Tverskaya, 4, 5) \\
1 \\
0 \\
1 \\
1 \\
\subsection*{Дневник отладки}
Проблем и ошибок при написании данной работы не возникло.

\subsection*{Недочёты}


\subsection*{Выводы}
В процессе выполнения работы я на практике познакомился с пользовательскими литералами. Это очень удобная и практическая вещь, о которой я не знал до курса ООП. Использование этого средства позволяет получать из заданных типов данных какие то данные, вычислять что то, без использования функций, а с помощью переопределения специального оператора



\vfill
\pagebreak
\subsection*{Исходный код:}

{\Huge adress.h}
\inputminted
    {C++}{adress.h}
    
{\Huge adress.cpp}
\inputminted
    {C++}{adress.cpp}
    
{\Huge main.cpp}
\inputminted
    {C++}{main.cpp}
    
\end{document}
